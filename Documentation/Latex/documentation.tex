% preamble %
\documentclass[12pt]{article}

% load preample %
\usepackage{amsfonts}
\usepackage{fancyhdr}
\usepackage{comment}
\usepackage[a4paper, top=1cm, bottom=1.5cm, left=2cm, right=2cm]{geometry}
\usepackage{enumitem}
\usepackage{times}
\usepackage{changepage}
\usepackage{amssymb}
\usepackage{graphicx}
\usepackage{tabularx}
\usepackage{titlesec}
\usepackage{hyperref}
\usepackage{changepage}
\usepackage[parfill]{parskip}
\usepackage{wrapfig}
\usepackage[export]{adjustbox}
\usepackage{multirow}
\usepackage{array}
\usepackage[table]{xcolor}
\usepackage{longtable,booktabs}
\usepackage{float}
\usepackage{listings} % For presenting code nicely
\usepackage{hyperref}
\usepackage{framed}

% settings %
\setcounter{secnumdepth}{2} % enumerate
\setcounter{tocdepth}{2}    % TOC entries
%\renewcommand{\contentsname}{Innholdsfortegnelse}
\newcounter{frcounter}
\newcounter{frsubcounter}[frcounter]
\newcounter{nfrcounter}
\newcounter{nfrsubcounter}[nfrcounter]
\titlespacing*{\paragraph}{\parindent}{1ex}{1em}

% commands %
    % counters %
\newcommand*{\FR}{\stepcounter{frcounter}\textbf{[FR-\arabic{frcounter}] \quad}}
\newcommand*{\FRsub}{\stepcounter{frsubcounter}\textbf{[FR-\arabic{frcounter}.\arabic{frsubcounter}] \quad}}
\newcommand*{\NFR}{\stepcounter{nfrcounter}\textbf{[NFR-\arabic{nfrcounter}] \quad}}
\newcommand*{\NFRsub}{\stepcounter{nfrsubcounter}\textbf{[NFR-\arabic{nfrcounter}.\arabic{nfrsubcounter}] \quad}}
\newcommand{\invis}{\phantom{a}}

    % requirement commands %
\newcommand*{\freq}[1]{\FR\textbf{#1}\\}
\newcommand*{\fsubreq}[1]{\FRsub{#1}\\}
\newcommand*{\nfreq}[1]{\NFR\textbf{#1}\\}
\newcommand*{\nfsubeq}[1]{\NFRsub{#1}\\}

    % colors %
\newcommand{\cellr}{\cellcolor{red!25}}
\newcommand{\cello}{\cellcolor{orange!25}}
\newcommand{\celly}{\cellcolor{yellow!25}}
\newcommand{\celll}{\cellcolor{lime!25}}
\newcommand{\cellg}{\cellcolor{green!25}}

\definecolor{pblue}{rgb}{0.13,0.13,1}
\definecolor{pgreen}{rgb}{0,0.5,0}
\definecolor{pred}{rgb}{0.9,0,0}
\definecolor{pgrey}{rgb}{0.46,0.45,0.48}
% \definecolor{clientCode}{HTML}{fff6db}
\definecolor{ccInner}{HTML}{edf3f5}
\definecolor{ccFrame}{HTML}{dce1e3}

\definecolor{shadecolor}{named}{ccFrame} 

% environments %
\newenvironment{subreq}{\begin{adjustwidth}{1cm}{}}{\end{adjustwidth}}

% lstset %
\lstset{
    backgroundcolor = \color{ccInner},
    language=Java,
    showspaces=false,
    showtabs=false,
    breaklines=true,
    showstringspaces=false,
    breakatwhitespace=true,
    commentstyle=\color{pgreen},
    keywordstyle=\color{pblue},
    stringstyle=\color{pred},
    basicstyle=\footnotesize\ttfamily, % font-size
    moredelim=[is][\textcolor{pgrey}]{\%}{\%}
}

% document %
\begin{document}
\title{%
    API Specification\\
    \large Silhouette}
\author{%
    Group 12:\\
    Mats Engelien\\
    Lars Erik Faber\\
    Håkon Marthinsen}
\date{}
\maketitle

\newpage

\tableofcontents

\newpage

\section{Intro}

\section{Group Description}

\subsection{Delegation of Work}

\subsubsection{Description of Work}
Each students writes about scenarios they have contributed with...
Workload...

\section{Background}
Establish existing solutions ...

\section{Design Specification}
High Level Design Principles...

\subsection{Design Patterns}
\subsubsection{Builder Pattern}
We were very quickly guided towards using the Builder pattern when we first had decided on the Framework-type. The Builder pattern, unlike the Abstract Factory pattern, is a lot more like a "Fluent interface", meaning it relies heavily on method cascading.
The builder pattern is useful as it attempts to seperate the construction of an object from the way it is represented, this way we can use the same process to create many different looking objects that are fundementally similar.
We need to use the builder pattern when we want immutable objects. We want immutable classes because they have a wide range of benefits when we are developing an API. They are simple to use, they are synchronization safe, we only need one constructor, and they are great for maps(which we're using) to name a few.

The main problem with implementing the builder pattern is the sheer amount of extra code required. The lines will most likely more than double. However, those extra lines of code give us much more readable code and design flexibility. We exchange parameters in the constructor for much prettier method calls.


\subsection{Design Decisions}

\subsection{Personal Decisions}

\section{Project Structure (File Structure)}

\subsection{Type Reference Documentation}
Link to type doc...

\section{Client Code}

\subsection{Scenarios and Solutions}

\subsubsection{Scenario 1}
Make two CSS rulesets and a grid. Give one of the rulesets a class to target and change the background-color of one ruleset. Add the grid to the same ruleset and set it's columns and rows. On the second ruleset, set the text-color, add it to the first ruleset, create a Container of type "header" and give it the same class as the ruleset.

\subsubsection{Scenario 1 - Proposed Solution}

\begin{lstlisting}
    RuleSet colorBlue = new RuleSet(".BlueClass");
    colorBlue.addRule("background-color", "blue");

    Grid blueGrid = new Grid();
    colorBlue.setColumns("10px", "12px");
    colorBlue.setRows("10px", "12px");

    RuleSet colorGreen = new RuleSet();
    colorGreen.addRule("font-color", Color.RGB(23, 52, 12));

    colorBlue.addRule(colorGreen);

    Container myHeader = new Container("header");
    myHeader.applyClass("BlueClass");

    -->
    <header class="BlueClass" ></header>
\end{lstlisting}


\subsubsection{Scenario 2}
Make a table whose size changes dynamically, add values to the header row and add values to the rest of the rows as they are generated. Apply a class to the table and set a header color for the table.

    \subsubsection{HTML}

    Scenario 1: Make a form that looks like a CV, with a picture, some general text informasjon about my self as a person (wiht age, name etc.) wher the informasjon holder is in a strong text and the informasjon about the person is just normal text.

    Scenario 2: Youtube copie, make a area for a video to be stord on the page where the name of the video is over the video area and make is so that it can be seen form that webpage and not a like to another wabpage.

    Scenario 3: Login and password input area

    Scenario 4: Questionnaire -> Make a form wiht several questions using radio and/or chekbox (inputs)

    \subsubsection{CSS}

    Scenario 1: Nav-bar with link(s) button / box around it that are flexible (when it comes to the length and height of the webpage), and when hover over it will change the background color.
        
    Scenario 2: Make stationary background so when some one scroll on the webpage the containers will comes form the top or bottom depending what way they are scrolling and overlap the background.

    Scenario 3: Make a webpage that has diffrent colored boxes that are normaly 3 in length on a normal pc screen and just 1 in length when its used on a phone.

\section{User Testing}

\subsection{Description of setup}

\subsection{The Code}

\subsection{Feedback}

\section{Revised API}

\section{Project Discussion and Conclusion}


\end{document}